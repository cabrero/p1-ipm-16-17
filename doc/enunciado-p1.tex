\documentclass[11pt,a4paper]{article}

\usepackage[spanish]{babel}
\usepackage[utf8]{inputenc}
\usepackage{url}


\title{Práctica 1 - Interfaces Gráficas de Usuaria con Gtk+ y Python}
\author{Interfaces Persona Máquina}
\date{Curso 16/17}

\renewcommand{\abstractname}{Objetivos}


\begin{document}


\maketitle

\begin{abstract}
  Aplicar los conocimientos adquiridos sobre el desarrollo de
  interfaces gráficas de usuaria creando una aplicación cuya interfaz
  gráfica de escritorio se implementará utilizando el lenguaje de
  programación Python y la librería gráfica Gtk+.
\end{abstract}


%%%%%%%%%%%%%%%%%%%%%%%%%%%%%%%%%%%%%%%%%%%%%%%%%%%%%%%%%%%%%%%%%%%%%%%%%%%
\section{Descripción}

La aplicación que desarrollarás será una pequeña base de datos de películas
que mostrará a la usuaria las películas que ya ha visto, las que tiene
pendientes de ver y, usando un servicio en red, ofrecerá recomendaciones
a partir de películas ya vistas.

Los siguientes apartados describen los \emph{sprints} que debes
realizar según el orden de realización.


\subsection{Requisitos no funcionales}
\begin{itemize}
\item La implementación se realizará con python y GTK+.
\end{itemize}


%%%%%%%%%%%%%%%%%%%%%%%%%%%%%%%%%%%%%%%%%%%%%%%%%%%%%%%%%%%%%%%%%%%%%%%%%%%
\section{Sprint 1}

En este \emph{sprint} debes realizar los siguientes pasos:

\begin{enumerate}
\item Diseña una IU que permita ver una lista de películas y realizar
las tareas básicas: añadir, borrar, editar.

  Puedes emplear el formato de tu elección para documentar el diseño.

\item Haz un diseño software de la aplicación para dar soporte a la
  interface que acabas de diseñar.

  El diseño debe ajustarse a la idea básica del MVC: separar la vista
  del modelo.

  Para documentar el diseño debes usar los diagramas UML necesarios.

\item Implementa el diseño.

\item Valida todos los pasos anteriores, en especial el funcionamiento
  de tu implementación. A continuación asígnale al último commit del
  repositorio la etiqueta \texttt{sprint1}.

\item Valida el contenido del repositorio remoto\footnote{HINT:
    después de clonarlo, puedes hacer un reset a la etiqueta
    \texttt{sprint1} (\texttt{git clone --hard sprint1})}.
\end{enumerate}



%%%%%%%%%%%%%%%%%%%%%%%%%%%%%%%%%%%%%%%%%%%%%%%%%%%%%%%%%%%%%%%%%%%%%%%%%%%
\section{Sprint 2}

En este \emph{sprint} debes realizar los siguientes pasos:

\begin{enumerate}
\item Internacionaliza la aplicación.

  Usa \texttt{gettext} para el soporte de idiomas.

\item Localiza el idioma de la aplicación a dos idiomas de tu
  preferencia.

\item Valida todos los pasos anteriores, en especial el funcionamiento
  de tu implementación. A continuación asígnale al último commit del
  repositorio la etiqueta \texttt{sprint2}.

\item Valida el contenido del repositorio remoto.
\end{enumerate}


%%%%%%%%%%%%%%%%%%%%%%%%%%%%%%%%%%%%%%%%%%%%%%%%%%%%%%%%%%%%%%%%%%%%%%%%%%%
\section{Sprint 3}

En este \emph{sprint} debes realizar los siguientes pasos:

\begin{enumerate}
\item Incrementa el diseño de la IU de forma que permita marcar las
  películas que ya hayan sido vistas, y poder ver por separado la
  lista de películas que han sido vistas y las que están pedientes.

\item Incrementa el diseño software de la aplicación para dar soporte
  a los cambios que estás introduciendo.

\item Implementa el diseño.

\item Valida todos los pasos anteriores, en especial el funcionamiento
  de tu implementación. A continuación asígnale al último commit del
  repositorio la etiqueta \texttt{sprint3}.

\item Valida el contenido del repositorio remoto.
\end{enumerate}



%%%%%%%%%%%%%%%%%%%%%%%%%%%%%%%%%%%%%%%%%%%%%%%%%%%%%%%%%%%%%%%%%%%%%%%%%%%
\section{Sprint 4}


En este \emph{sprint} debes realizar los siguientes pasos:

\begin{enumerate}
\item Incrementa el diseño de la IU para incluir un mecanismo que
  ofrezca a la usuaria recomendaciones a partir de las películas ya
  vistas.

  IMPORTANTE: Las recomendaciones se obtienen de un servicio en red.

\item Incrementa el diseño software de la aplicación para dar soporte
  a los cambios que estás introduciendo.

\item Implementa el diseño.

  Puedes usar el servicio en red de tu preferencia. Si no conoces ninguno,
  te sugerimos el ofrecido en \url{https://www.themoviedb.org/}

\item Valida todos los pasos anteriores, en especial el funcionamiento
  de tu implementación. A continuación asígnale al último commit del
  repositorio la etiqueta \texttt{sprint4}.

\item Valida el contenido del repositorio remoto.
\end{enumerate}





%%%%%%%%%%%%%%%%%%%%%%%%%%%%%%%%%%%%%%%%%%%%%%%%%%%%%%%%%%%%%%%%%%%%%%%%%%%
\section{Sprint 5}

En este \emph{sprint} debes realizar los siguientes pasos:

\begin{enumerate}
\item Documenta y corrige los casos en que la interface no cumple el
  principio ``principle of least astonishment''.

\item Documenta y corrige los casos en que la interface:
  \begin{itemize}
  \item no gestiona los errores,
  \item no proporciona \textit{feedback} cuando es necesario,
  \item no gestiona la concurrencia, i.e. se bloquea.
  \end{itemize}

\item Documenta y corrige los casos en que la interface no cumple las
  \emph{Gnome Human Interface Guidelines}.

\item Valida todos los pasos anteriores, en especial el funcionamiento
  de tu implementación. A continuación asígnale al último commit del
  repositorio la etiqueta \texttt{sprint5}.

\item Valida el contenido del repositorio remoto.
\end{enumerate}



\end{document}
